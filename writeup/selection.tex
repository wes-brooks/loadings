\section{Modeling}

\subsection{Rainfall-driven events}
We begin with a discussion of the loading events that were driven by rainfall.

\subsection{Variable Selection}
In order to make a model of the load carried by the stream, we need to select the predictor variables that have explanatory power. We use stepwise regression with the Bayesian Information Criterion (BIC) to screen the potential predictor variables. Those variables that survive the screening are then added to a linear model that describes the logarithm of the total load carried during each event. The table shows how the model fit (in terms of $\text{R}^2$) improves as additional predictor variables are included.\\

The variables are ranked in the order of their influence on the output, where a variable's influence is defined as the standard deviation of the observed values of that variable, multiplied by its  coefficient in the model. In all cases, the amount of rainfall during an event (theissen) was the most important predictor of the load. At all sites except Brewery Creek, the next most important predictor was the antecedent base flow, which we consider a proxy for the average soil moisture in the watershed. When the ground is already sodden, additional rain will not infiltrate as quickly and is more likely to produce erosive runoff.\\

In the case of Brewery Creek, the antecedent base flow was not an important predictor of runoff. Instead, we saw the peak 30-minute intensity of rainfall show up as an important predictor.\\

