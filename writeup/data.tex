\section{Data}


\subsection{Description}
The data in this report comes from four Wisconsin streams that were monitored (with some gaps in data collection) between 1989 and 2007. The streams and the period during which each was monitored are:\\


\begin{table}[c]
    \begin{tabular}[r|l]
        Stream & Monitored \\
        \hline
        Eagle Creek & 1991-1994, 2003-2007\\
        Joos Valley Creek & 1990-1994, 2002-2007\\
        Otter Creek & 1990-1997, 2000-2002\\
        Brewery Creek & 1989, 1994-2002, 2004-2005\\
    \end{tabular}
\end{table}


Each entry in our data set represents one "loading event", which is defined based on the hydrograph - the event begins when the loading rises from a base level toward a peak, and ends when the loading falls back to its new base level. Typically, the loading rises during and after a storm, when rainfall erodes the soil and washes sediment into the stream. Other times, sediment is carried by melting snow.\\


Different potential predictor variables are measured during the two different categories of event (rainfall-driven and snowmelt-driven). Data on the amount of rainfall is only collected when the ground is free of snow because snow interferes with the rain gauges. Data on the amount of snowmelt during an event is only available during some of the snowmelt-driven events (e.g. we cannot estimate the amount of snow that melted when there was additional snow falling at the same time). Because of this, the two categories are modeled separately.\\


Within the category of events that are driven by rainfall, we investigate making a further split (on May 15 each year) between "early-season" and "late-season" events. Erosion may be more common early in the spring, before most of the summer's vegetation appears, which would alter the relationship between our inputs and outputs. See the Modeling section for further discussion of this split.\\



\subsection{Exploratory Analysis}
Our anaysis targets two outputs: the phosphorus and sediment loads carried by each stream. Using "Rainmaker" software, each load can be broken into two parts: base load and storm load. We will consider models of the storm load and of the total load.\\


Over the course of the monitoring period, the majority of the total load (both of sediment and of phosphorus) was carried during just a few major events. Just 10\% of the events carried between \Sexpr{100*round( min(q_90$stot_tot), 1 )}\% (at \Sexpr{names(q_90$stot_tot)[q_90$stot_tot==min(q_90$stot_tot)]}) and \Sexpr{100*round( max(q_90$stot_tot), 1 )}\% (at \Sexpr{names(q_90$stot_tot)[q_90$stot_tot==max(q_90$stot_tot)]}) of the total sediment load and \Sexpr{100*round( min(q_90$ptot_tot), 1 )}\% (at \Sexpr{names(q_90$ptot_tot)[q_90$ptot_tot==min(q_90$ptot_tot)]}) and \Sexpr{100*round( max(q_90$ptot_tot), 1 )}\% (at \Sexpr{names(q_90$ptot_tot)[q_90$ptot_tot==max(q_90$ptot_tot)]}) of the total phosphorus load.\\


These major events occured during all three annual periods, roughly in proportion to the amount of loading during that period...
