\section{Data}
The sediment load carried in a stream is not constant in time; a recent study \cite{Danz:2010} has shown that most of the annual load is carried in just a few days when the stream is heavily loaded with sediment. The same is true for the phosphorus load. Our interest is in modeling the load carried by a stream based on some simple predictors. In order to mitigate the harmful effects of erosion and sediment runoff, watershed managers need an accurate idea of how inputs like rainfall and snowmelt affect the amount of sediment carried in the stream.\\

Each entry in our data set represents one "loading event", which is defined based on the hydrograph - the event begins when the loading rises from a base level toward a peak, and ends when the loading falls back to its new base level. Typically, the loading rises during and after a storm, when rainfall erodes the soil and washes sediment into the stream. Other times, sediment is carried by melting snow.\\

Different potential predictor variables are measured during the two different categories of event (rainfall-driven and snowmelt-driven). Data on the amount of rainfall is only collected when the ground is free of snow because snow interferes with the rain gauges. Data on the amount of snowmelt during an event is only available during some of the snowmelt-driven events (e.g. we cannot estimate the amount of snow that melted when there was additional snow falling at teh same time). Because of this, the two categories are modeled separately.\\

Within the category of events that are driven by rainfall, we investigate making a further split (on May 15 each year) between "early-season" and "late-season" events. Erosion may be more common early in the spring, before most of the summer's vegetation appears, which would alter the relationship between our inputs and outputs. See the Modeling section for further discussion of this split.\\